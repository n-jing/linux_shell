\documentclass[12pt, UTF8]{article} % 12pt 为字号大小 UTF8
\usepackage{amssymb,amsfonts,amsmath,amsthm}

\usepackage{tgbonum}

%\usepackage{fontspec,xltxtra,xunicode}
%\usepackage{times}

%----------
% 定义中文环境
%----------

\usepackage{xeCJK}
\usepackage{zhnumber}
% \setCJKmainfont[BoldFont={SimHei},ItalicFont={KaiTi}]{SimSun}
% \setCJKsansfont{SimHei}
% \setCJKfamilyfont{zhsong}{SimSun}
% \setCJKfamilyfont{zhhei}{SimHei}

% \newcommand*{\songti}{\CJKfamily{zhsong}} % 宋体
% \newcommand*{\heiti}{\CJKfamily{zhhei}}   % 黑体


%----------
% 版面设置
%----------
%首段缩进
\usepackage{indentfirst}
\setlength{\parindent}{2.1em}

%行距
\renewcommand{\baselinestretch}{1.4} % 1.4倍行距

%页边距
\usepackage[a4paper]{geometry}
\geometry{verbose,
  tmargin=3cm,% 上边距
  bmargin=3cm,% 下边距
  lmargin=3cm,% 左边距
  rmargin=3cm % 右边距
}


%----------
% 其他宏包
%----------
%图形相关
\usepackage[x11names]{xcolor} % must before tikz, x11names defines RoyalBlue3
\usepackage{graphicx}
\usepackage{pstricks,pst-plot,pst-eps}
\usepackage{subfig}
\def\pgfsysdriver{pgfsys-dvipdfmx.def} % put before tikz
\usepackage{tikz}

%原文照排
\usepackage{verbatim}

%网址
\usepackage{url}

%----------
% 习题与解答环境
%----------
% %习题环境
% \theoremstyle{definition} 
% \newtheorem{exs}{习题}

% %解答环境
% \ifx\proof\undefined\
% \newenvironment{proof}[1][\protect\proofname]{\par
% \normalfont\topsep6\p@\@plus6\p@\relax
% \trivlist
% \itemindent\parindent
% \item[\hskip\labelsep
% \scshape
% #1]\ignorespaces
% }{%
% \endtrivlist\@endpefalse
% }
% \fi

% \renewcommand{\proofname}{\it{证明}}

%----------
% 我的自定义
%----------

\newcommand{\horrule}[1]{\rule[0.5ex]{\linewidth}{#1}} 	% Horizontal rule

\renewcommand{\refname}{参考文献}
\renewcommand{\abstractname}{\large \bf 摘\quad 要}
\renewcommand{\contentsname}{目录}
\renewcommand{\tablename}{表}
\renewcommand{\figurename}{图}


\setlength{\parskip}{0.4ex} % 段落间距

\usepackage{enumitem}
\setenumerate[1]{itemsep=0pt,partopsep=0pt,parsep=\parskip,topsep=5pt}
\setitemize[1]{itemsep=0.4ex,partopsep=0.4ex,parsep=\parskip,topsep=0.4ex}
\setdescription{itemsep=0pt,partopsep=0pt,parsep=\parskip,topsep=5pt}


%==========
% 正文部分
%==========

\begin{document}

\title{
%% {\normalfont\normalsize\textsc{
%% Nanjing University of Aeronautics and Astronautics\\
%% Course Name, Autumn 2017 \\[25pt]}}
%% \horrule{0.5pt}\\
\sffamily{中文报告模板}
%% \horrule{1.8pt}\\[20pt]
}
\author{蒋静\\ \sffamily{siliuhe@sina.com}}
\date{\zhtoday} % 若不需要自动插入日期,则去掉前面的注释;{ } 中也可以自定义日期格式

%% \begin{titlepage}
\maketitle
%% \vspace{30pt}
%% \begin{abstract}
%% \normalsize \ \ 这是中文摘要。大概写满这一页可以了。摘要又称概要、内容提要。摘要是以提供文献内容梗概为目的,不加评论和补充解释,简明、确切地记述文献重要内容的短文。其基本要素包括研究目的、方法、结果和结论。具体地讲就是研究工作的主要对象和范围,采用的手段和方法,得出的结果和重要的结论,有时也包括具有情报价值的其它重要的信息。\\[5pt]
%% \indent \ \ \textbf{关键词}:图卷积神经网络,复杂网络,表示学习
%% \end{abstract}
%% \thispagestyle{empty}
%% \end{titlepage}

%% \tableofcontents
%% \thispagestyle{empty}

%% \newpage
\setcounter{page}{1}

\section{模型}
正文是指著作的本文,有规范格式和生效标志的正式文本。

\subsection{字体}
默认字体为宋体。{\sffamily 这是黑体。} {\rmfamily 这是宋体。} {\ttfamily 这是仿宋。} {\it 这是楷体。}
或者\textsf{黑体},\textrm{宋体},\texttt{仿宋},\textit{楷体}。

\subsubsection{文字强调}
加粗宋体:\textbf{粗体},加斜字体自动变成楷体:\textit{强调}。

更多中文说明(网址有点长,显示不全):\\\url{https://www.overleaf.com/latex/examples/using-the-ctex-package-on-overleaf-zai-overleafping-tai-shang-shi-yong-ctex/gndvpvsmjcqx/viewer.pdf}

\subsection{图}
\begin{figure}[ht]
\centering
%% \includegraphics[width=\textwidth]{canoform.png}
\caption{这是一个图}
\label{fig:fig1}
\end{figure}
引用图\ref{fig:fig1} 。

\subsection{表}
\begin{table}[ht]
\caption{这是一个表}
\label{tb:filter}
\centering
\begin{tabular}{cccc}
\hline
 & 卡尔曼滤波 & 神经网络滤波 & 被动无源滤波 \\ 
\hline
模型类型 & 线性 & 线性 & 非线性 \\ 
参数调校 & 大量 & 几乎没有 & 合理 \\ 
稳定性 & 满足全局稳定性 & 依赖于模型 & 满足子系统稳定性 \\ 
\hline
\end{tabular} 
\end{table}
引用表格\ref{tb:filter} 。

%% \begin{thebibliography}{9}

%% \addcontentsline{toc}{section}{参考文献}  % 目录中加入参考文献

%% \bibitem{cao17}
%%   Rongmei Cao.
%%   Matrix Theory.
%%   Nanjing University of Aeronautics and Astronautics, 2017.  

%% \bibitem{pbrs14}
%%   Perozzi, Bryan, R. Al-Rfou, and S. Skiena. "DeepWalk: online learning of social representations." (2014):701-710.

%% \bibitem{czw17}
%%   李彦冬, 郝宗波, and 雷航. "卷积神经网络研究综述." 计算机应用 36.9(2016):2508-2515.

%% \end{thebibliography}

\end{document}
 
